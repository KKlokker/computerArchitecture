\documentclass[12pt, a4paper]{article}
\usepackage{caption}
\usepackage{graphicx}
\usepackage{hyperref}
\hypersetup{
    colorlinks,
    citecolor=black,
    filecolor=black,
    linkcolor=black,
    urlcolor=black
} 

\usepackage{tikz-network}
\usepackage{amsmath, amsfonts, amssymb, amsthm}
\usepackage{algpseudocode}
\usepackage{algorithm}
\title{Computer Architecture\\and system programming}
\date{2022}
\author{Kristoffer Klokker}

\usepackage{xcolor,listings}
\usepackage{textcomp}
\usepackage{color}
\usepackage{listings}
\definecolor{codegreen}{rgb}{0,0.6,0}
\definecolor{codegray}{rgb}{0.5,0.5,0.5}
\definecolor{codepurple}{HTML}{C42043}
\definecolor{backcolour}{HTML}{F2F2F2}
\definecolor{bookColor}{cmyk}{0,0,0,0.90}  
\color{bookColor}

\lstset{upquote=true}

\lstdefinestyle{mystyle}{
    backgroundcolor=\color{backcolour},   
    commentstyle=\color{codegreen},
    keywordstyle=\color{codepurple},
    numberstyle=\numberstyle,
    stringstyle=\color{codepurple},
    basicstyle=\footnotesize\ttfamily,
    breakatwhitespace=false,
    breaklines=true,
    captionpos=b,
    keepspaces=true,
    numbers=left,
    numbersep=10pt,
    showspaces=false,
    showstringspaces=false,
    showtabs=false,
    tabsize=3,
}


\lstset{style=mystyle}
\usepackage{zref-base}

\makeatletter
\newcounter{mylstlisting}
\newcounter{mylstlines}
\lst@AddToHook{PreSet}{%
  \stepcounter{mylstlisting}%
  \ifnum\mylstlines=1\relax
    \lstset{numbers=none}
  \else
    \lstset{numbers=left}
  \fi
  \setcounter{mylstlines}{0}%
}
\lst@AddToHook{EveryPar}{%
  \stepcounter{mylstlines}%
}
\lst@AddToHook{ExitVars}{%
  \begingroup
    \zref@wrapper@immediate{%
      \zref@setcurrent{default}{\the\value{mylstlines}}%
      \zref@labelbyprops{mylstlines\the\value{mylstlisting}}{default}%
    }%
  \endgroup
}

% \mylstlines print number of lines inside listing caption
\newcommand*{\mylstlines}{%
  \zref@extractdefault{mylstlines\the\value{mylstlisting}}{default}{0}%
}
\makeatother


\newcommand\numberstyle[1]{%
    \footnotesize
    \color{codegray}%
    \ttfamily
    \ifnum#1<10 0\fi#1 |%
}


\begin{document}
	\maketitle
	\clearpage
	\tableofcontents
	\clearpage
	\section{Exercises}
		\subsection{Benchmark}
			For the 40MHz processor which performed the instructions\\[4mm]
			\begin{tabular}{c c c}
				 \hline
				 Instruciton type & Instruction count & Cycles per instruction\\
				 \hline
				 Integer artihemtic & 41,000 & 1\\
				 Data transfer & 28,000 & 2\\
				 Floating point & 25,000 & 2\\
				 Control transfer & 6,000 & 2\\
				 \hline
			\end{tabular}
			\subsubsection{Find the average CPI}
				$$\frac{1\cdot 41000+2\cdot 28000+2\cdot 25000 + 2\cdot 6000}{100000}=1.59$$
				CPI is the average cycles pr instruction. Therefore 4.5
			\subsubsection{Execution time}
				\begin{align*}
					CPI&=1.59\\
					I_c&=100000\\
					\tau&=\frac{1}{f}=\frac{1}{40000000Hz}\\
					 T&=I_c\cdot CPI \cdot \tau \\
					 &= 1.59\cdot 100000\cdot \frac{1}{40000000Hz}\\
					 T&=0.003975s
				\end{align*}
			\subsubsection{MIPS}
				\begin{align*}
					MIPS=\frac{f}{CPI\cdot 10^6}\\
					CPI&=1.59\\
					f=40000000Hz\\
					MIPS=\frac{40000000Hz}{1.59\cdot 10^6}\\
					MIPS=25.16\frac{1}{s}
				\end{align*}
		\subsection{Explain how a negative number is represnted in the following representation}
			\begin{itemize}
				\item Sign-magnitude - The left most bit must be 1 which result in the rest being interpretated as negative
				\item Twos compliment - The left most bit is 1 to subtract the maximum value from the rest
				\item Biased - Bias is most usually half the range therefore a negative value is simply less half the maximum value
			\end{itemize}
		\subsection{Represent the following in 8 bit twos compliment and sing magnitude}
			\begin{itemize}
				\item 64 - 00100000
				\item -28 - twos 11100100 - sign 10011100
			\end{itemize}
		\subsection{Convert from twos compliment to decimal}
			\begin{itemize}
				\item 1100110 : -26
				\item 1011101 : -35
			\end{itemize}
		\subsection{Show the calculations in 8 bit twos compliment}
			\subsubsection{6+12}
				\begin{align*}
					6=00000110\\
					12=00001100\\
					00000110+00001100=00010010
				\end{align*}
			\subsubsection{-6+12}
				\begin{align*}
					-6=11111010\\
					12=00001100\\
					11111010+00001100=00000110
				\end{align*}
				Overflow is ignored
			\subsubsection{6-12}
				\begin{align*}
					6=00000110\\
					-12=11110100\\
					00000110+11110100=11111010
				\end{align*}
			\subsubsection{-6-12}
				\begin{align*}
					-6=11111010\\
					-12=11110100\\
					11111010+11110100=11101110
				\end{align*}
				Overflow is ignored
		\subsection{Fill out the table for the most twos compliment addition}
				\begin{table}[h!]
				\begin{tabular}{ccc|ccc}
				\hline
				          & Input     &           &           & Output    &   \\ \hline
				$x_{n-1}$ & $y_{n-1}$ & $c_{n-2}$ & $z_{n-1}$ & $c_{n-1}$ & v \\ \hline
				0         & 0         & 0         & 0         & 0         & 0 \\
				0         & 0         & 1         & 0         & 0         & 1 \\
				0         & 1         & 0         & 1         & 0         & 0 \\
				0         & 1         & 1         & 0         & 1         & 0 \\
				1         & 0         & 0         & 1         & 0         & 0 \\
				1         & 0         & 1         & 0         & 1         & 0 \\
				1         & 1         & 0         & 1         & 1         & 0 \\
				1         & 1         & 1         & 1         & 1         & 1 \\ \hline
				\end{tabular}
				\end{table}
				Here $x_{n-1}$ and $y_{n-1}$ is the most signifact bits of the two addends.\\
				$c$ is the carry bit and $z_{n-1}$ is the results most significatn bit. \\
				$v$ is a bit singnaling overflow.\\
				If can be seen in row 2 and the last row that:\\
				Overflow occurs if and only if the carry into the addition of the MSBs isdifferent from the carry out of that addition
		\subsection{Convert 23 and 29 to 6 bit twos compliment and multiply using Booths algorithm}
				\begin{align*}
					23=010111\\
					29=011101\\
					A=0\\
					Q_{-1}=0\\
					M=010111\\
					Q=011101\\
					count=5\\[4mm]
					Q_0,Q_{-1}=10\\
					A=A-M=101001\\
					shift\; A=101001\; Q=011101\; Q_{-1}=0\\
					A=110100\\
					Q=101110\\
					Q_{-1}=1\\
					count=4\\[4mm]
					Q_0,Q_{-1}=01\\
					A=A+M=110100+010111=001011\\
					shift\\
					A=000101\\
					Q=110111\\
					Q_{-1}=0\\
					count=3\\[4mm]
					Q_0,Q_{-1}=10\\
					A=A-M=000101-011001\\
					A=000101+101001=101110\\
					shift\; A=101110\; Q=110111\; Q_{-1}=0\\
					A=110111\\
					Q=011011\\
					Q_{-1}=1\\
					count=3
				\end{align*}
				\begin{align*}
					Q_0,Q_{-1}=11\\
					shift\; A=110111\; Q=011011\; Q_{-1}=1\\
					A=111011\\
					Q=101101\\
					Q_{-1}=1\\
					count=2\\[4mm]
					Q_0,Q_{-1}=11\\
					shift\; A=111011\; Q=101101\; Q_{-1}=1\\
					A=111101\\
					Q=110110\\
					Q_{-1}=1\\
					count=1\\[4mm]
					Q_0,Q_{-1}=01\\
					A=A+M=111101+011001\\
					A=010100\\
					shift\; A=010100\; Q=110110\; Q_{-1}=1\\
					A=001010\\
					Q=011011\\
					Q_{-1}=0\\
					count=0\\[4mm]
					010111\times 011101 = AQ=001010011011
					\end{align*}
		
\end{document}
